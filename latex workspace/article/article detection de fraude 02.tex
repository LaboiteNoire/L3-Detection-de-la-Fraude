\documentclass[10pt,a4paper]{article}
\title{Détection de la Fraude}
\author{YOUSSFI, Hanane\\ 
		\texttt{hanane.youssfi@etu.univ-lyon1.fr}
  		\and
  		GOGUY, Priscilla\\
  		\texttt{priscilla.goguy@etu.univ-lyon1.fr}
  		\and
  		ZROUAMA, Emmanuel\\
  		\texttt{kaussaha-emmanuel-re.zrouama@etu.univ-lyon1.fr}
  		\and
  		REINETTE, Mahlî\\
  		\texttt{mahli.reinette@etu.univ-lyon1.fr}}
\date{}
\usepackage[utf8]{inputenc}
\usepackage[T1]{fontenc}
\usepackage[french]{babel}
\usepackage{graphicx}
\usepackage{hyperref}


\usepackage{listings}
\usepackage{xcolor}


\usepackage[many]{tcolorbox}
\usepackage{lipsum}
\usepackage{tikz}
\usetikzlibrary{automata, positioning, arrows}
\usepackage{pgfplots}
\usepackage{listings,xcolor}
\usepackage{tzplot}



% le préambule











% initialisation des couleurs
\definecolor{codegreen}{rgb}{0,0.6,0}
\definecolor{codegray}{rgb}{0.5,0.5,0.5}
\definecolor{codepurple}{rgb}{0.58,0,0.82}
\definecolor{backcolour}{rgb}{0.95,0.95,0.92}






%block de code 
\lstdefinestyle{mystyle}{
    backgroundcolor=\color{backcolour},   
    commentstyle=\color{codegreen},
    keywordstyle=\color{magenta},
    numberstyle=\tiny\color{codegray},
    stringstyle=\color{codepurple},
    basicstyle=\ttfamily\footnotesize,
    breakatwhitespace=false,         
    breaklines=true,                 
    captionpos=b,                    
    keepspaces=true,                 
    numbers=left,                    
    numbersep=5pt,                  
    showspaces=false,                
    showstringspaces=false,
    showtabs=false,                  
    tabsize=2
}

\lstset{style=mystyle}






%initialisation graphics
\tikzset{
->,
node distance = 2cm}





\lstdefinestyle{mystyle}
{
    style=Common,
    backgroundcolor=\color{Black},
    basicstyle=\scriptsize\color{White}\ttfamily,
    keywordstyle=\color{Orange},
    identifierstyle=\color{Cyan},
    stringstyle=\color{Red},
    commentstyle=\color{Green}
}

%\lstset{style=mystyle}













% le corps du document






% le préambule
\begin{document}
%titre
\pagecolor{black!20}
\maketitle


\begin{center}
	\includegraphics[width=4cm,height=2cm]{img9}
\end{center}


\newpage
\tableofcontents

\part{Scores Marginaux}
\subsection{Introduction}

\subsection{Pondération}


\part{Userform}
\begin{center}
\begin{minipage}[r]{0.1\textwidth}

\end{minipage}
\begin{minipage}[r]{0.8\textwidth}

Cette seconde partie est consacrée a la création d'interfaces graphique, permettant de communiquer avec nos différentes bases de données, de signaler les potentiels 

\end{minipage}
\end{center}
\section{Liste Portefeuille}

\UseRawInputEncoding\begin{lstlisting}
Private Sub UserForm_Initialize()
    Dim num_col As Integer
    Dim num_lin As Integer
    num_col = 14
    num_lin = 1
    With Me.ListView1
        .ListItems.Clear
        .Gridlines = True
        With .ColumnHeaders
            .Clear
            .Add 1, , "id", 30
            .Add 2, , "caractere frauduleux", 60
            .Add 3, , "nom", 40
            .Add 4, , "age", 25
            .Add 5, , "date souscription", 50
            .Add 6, , "type sinistre", 50
            .Add 7, , "lieu", 50
            
            .Add 8, , "date sinistre", 50
            .Add 9, , "date déclaration", 60
            .Add 10, , "météo", 40
            .Add 11, , "déscription", 100
            .Add 12, , "cout", 25
            .Add 13, , "usage", 30
            .Add 14, , "crm", 25
            .Add 15, , "derniere consultation", 70
        End With
        While Worksheets("Sheet1").Cells(num_lin + 1, 1) <> ""
            .ListItems.Add , , Worksheets("Sheet1").Cells(num_lin + 1, 1)
            .ListItems(num_lin).ListSubItems.Add , , Worksheets("Sheet1").Cells(num_lin + 1, 17)
            Dim j As Integer
            For j = 2 To num_col
                .ListItems(num_lin).ListSubItems.Add , , Worksheets("Sheet1").Cells(num_lin + 1, j)
            Next j
            num_lin = num_lin + 1
        Wend
    End With
    Worksheets("options").Cells(2, 1) = num_col
    Worksheets("options").Cells(2, 2) = num_lin
    Dim num_lin_reg As Integer
    num_lin_reg = 1
    
    With Me.ListView2
        .ListItems.Clear
        .Gridlines = True
        With .ColumnHeaders
            .Clear
            .Add 1, , "id", 30
            .Add 2, , "score", 40
            .Add 3, , "nom", 40
            .Add 4, , "age", 25
            .Add 5, , "date souscription", 50
            .Add 6, , "type sinistre", 50
            .Add 7, , "lieu", 50
            
            .Add 8, , "date sinistre", 50
            .Add 9, , "date déclaration", 60
            .Add 10, , "météo", 40
            .Add 11, , "déscription", 100
            .Add 12, , "cout", 25
            .Add 13, , "usage", 30
            .Add 14, , "crm", 25
            .Add 15, , "derniere consultation", 70
        End With
        While Worksheets("register").Cells(num_lin_reg + 1, 1) <> ""
            .ListItems.Add , , Worksheets("register").Cells(num_lin_reg + 1, 1)
            .ListItems(num_lin_reg).ListSubItems.Add , , Worksheets("register").Cells(num_lin_reg + 1, 16)
            Dim v As Integer
            For v = 2 To num_col
                .ListItems(num_lin_reg).ListSubItems.Add , , Worksheets("register").Cells(num_lin_reg + 1, v)
            Next v
            num_lin_reg = num_lin_reg + 1
        Wend
    End With
    Worksheets("options").Cells(2, 3) = num_lin_reg
    
End Sub
\end{lstlisting}
%passage en utf-8
\inputencoding{utf8}

\section{Fenêtres Client}

\section{Enregistrement de nouveaux sinistres}

\section{Choix final}

\end{document}
