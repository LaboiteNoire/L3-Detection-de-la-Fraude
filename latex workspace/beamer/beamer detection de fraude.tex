\documentclass[10pt]{beamer}
\title{Détection de la Fraude}
\author{YOUSSFI Hanane, GOGUY Priscilla, ZROUAMA Emmanuel, REINETTE Mahlî}
\date{}
\usepackage[utf8]{inputenc}
\usepackage[T1]{fontenc}
\usepackage[french]{babel}
\usepackage{graphicx}
\usepackage{hyperref}
\usepackage{amsmath}
\usepackage{xcolor}
\usepackage{libertine}


\usepackage{listings}
\usepackage{xcolor}


\usepackage[many]{tcolorbox}
\usepackage{lipsum}
\usepackage{tikz}
\usetikzlibrary{automata, positioning, arrows}
\usepackage{pgfplots}


\usetheme{Berlin}
% le préambule











% initialisation des couleurs
\definecolor{codegreen}{rgb}{0,0.6,0}
\definecolor{codegray}{rgb}{0.5,0.5,0.5}
\definecolor{codepurple}{rgb}{0.58,0,0.82}
\definecolor{backcolour}{rgb}{0.95,0.95,0.92}








%initialisation graphics
\tikzset{
->,
node distance = 2cm}












\lstdefinestyle{mystyle}{
    backgroundcolor=\color{backcolour},   
    commentstyle=\color{codegreen},
    keywordstyle=\color{magenta},
    numberstyle=\tiny\color{codegray},
    stringstyle=\color{codepurple},
    basicstyle=\ttfamily\footnotesize,
    breakatwhitespace=false,         
    breaklines=true,                 
    captionpos=b,                    
    keepspaces=true,                 
    numbers=left,                    
    numbersep=5pt,                  
    showspaces=false,                
    showstringspaces=false,
    showtabs=false,                  
    tabsize=2
}

\lstset{style=mystyle}







%numérotation
%\setbeamertemplate{footline}[frame number]





\begin{document}

\begin{frame}
\maketitle
\end{frame}



\begin{frame}
\frametitle{Table des matières}
\tableofcontents[]
\end{frame}












\section{Introduction}
\subsection{introduction}
\begin{frame}
\frametitle{Introduction}

\footnotemark[1]
\footnotetext[1]{}
\end{frame}


\begin{frame}
\frametitle{Plan}
\begin{enumerate}
	\item 1
	\item 2
	\item 3
	\item 4
\end{enumerate}
\end{frame}











\begin{frame}
\begin{center}
%definition 02
\begin{tcolorbox}[enhanced,title=vide,attach boxed title to top left=
{xshift=-2mm,yshift=-2mm},rightrule=1mm,colframe=blue!75!black,colbacktitle=blue!85!black]
contenu
\bsc{exemple de matrice : }
\begin{displaymath}
	\sigma = \bigl(\begin{matrix}
    u_1 & u_2 & \cdots & u_n \\
    v_2 & v_2 & \cdots & v_n
  \end{matrix}\bigr)
  .
\end{displaymath}

\tcblower
\begin{displaymath}
	e : 
	\begin{cases}
		\text{$\Sigma_1^* \to (\Sigma_2^*$ + $\Sigma_1^*)$}\\
		m \longmapsto 
			\begin{cases}
			e(m_a).v_i.e(m_b) \text{ si } m = m_a.u_i.m_b \\
			m \text{ sinon}
			\end{cases}
	\end{cases}
	.
\end{displaymath}
\end{tcolorbox} 
\end{center}
\end{frame}












\begin{frame}
\frametitle{Exemple}
\begin{tcolorbox}[enhanced,title=exemple,attach boxed title to top left=
{xshift=-2mm,yshift=-2mm},colback=yellow!10,colbacklower=blue,rightrule=1mm]
\begin{center}
\begin{tikzpicture}[auto,shorten >=1pt,node distance=2cm,initial text=,scale=0.5, transform shape]

	\node[state,initial]  (q_0) {$q_0$};
	\node[state,above right=of q_0] (q_1) {$q_1$};
	\node[state,below right=of q_0] (q_2) {$q_2$};
	\node[state,accepting,right=of q_1] (q_3) {$q_f$};
	\node[state,accepting,right=of q_2] (q_4) {$q_f$};
	
	
	\draw (q_0) edge node[above] {$a$} (q_1);
	\draw (q_0) edge node[above] {$aa$} (q_2);
	\draw (q_1) edge node[above] {$a$ $\longmapsto$ $c$} (q_3);
	\draw (q_2) edge node[below] {$aa$ $\longmapsto$ $b$} (q_4);
\end{tikzpicture}
\end{center}
\end{tcolorbox}
\end{frame}









\begin{frame}
\begin{minipage}[t]{0.95\textwidth}
\begin{tcolorbox}[enhanced,title=vide,attach boxed title to top left=
{xshift=-2mm,yshift=-2mm},rightrule=1mm,colframe=blue!75!black,colbacktitle=blue!85!black]

\begin{displaymath}
	f' \text{ } : \text{ } 
	\begin{cases}
		\text{$S \times \Sigma^* \to $ $P(E)$} \\
		(s,m) \longmapsto 
			\begin{cases}
				f(s,m) \text{ si } m \in E_{f(s,m)}\\
				\emptyset \text{ sinon}
			\end{cases}
	\end{cases}
\end{displaymath}
\tcblower
\begin{displaymath}
\begin{tikzpicture}[auto,shorten >=1pt,node distance=2cm,initial text=,scale=0.8, transform shape]
	\node[state,initial]  (q_0) {$q_0$};
	\node[state,right=of q_0] (q_1) {$q_1$};
	\node[state,accepting,right=of q_1] (q_2) {$q_f$};
	
	\draw (q_0) edge node[above] {$m_1 \in E_1$} (q_1);
	\draw (q_1) edge node[above] {$m_2 \in E_2$} (q_2);
\end{tikzpicture}
\end{displaymath} .

\end{tcolorbox}
\end{minipage}
\end{frame}








\end{document}

